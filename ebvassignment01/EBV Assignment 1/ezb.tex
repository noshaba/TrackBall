\documentclass{ezb}
\usepackage[]{todonotes}
\usepackage{longtable}
\usepackage{booktabs}

\renewcommand{\thesubsection}{\alph{subsection}}
\begin{document}

% \maketitle{Nummer}{Abgabedatum}{Tutor-Name}{Gruppennummer}
%           {Teilnehmer 1}{Teilnehmer 2}{Teilnehmer 3}
\maketitle{1}{\today}{U. Frese}{}
          {A.O.}{Frank Ihle - 3010158}{Simon Schirrmacher}{N.C.}

%-------Text-Start------------------------------------------
\section{Wer hat mein Tellerchen verrückt? (10 Punkte)}

\newpage
\section{Kekse (4 Punkte)}
Mit der Hilfe der Bildverarbeitung soll eine optische Qualitätskontrolle von einem Butterkeks durchgeführt werden, damit dem Kunde keine Ware mit fehlenden Zähnen oder mit anderen Verunstaltungen ausgeliefert wird. Abb. \ref{fig:keks_original} zeigt ein Beispiel von einem Keks, wie es der Konsument erhalten soll.
\begin{figure}[!h]
\begin{center}
    \includegraphics[scale=0.3]{Keks_original.png}
\end{center}
    \caption{Beispiel für einen Keks ohne fehlender Zähne.}
    \label{fig:keks_original}
\end{figure}\\
\newline
\textbf{{\large Fehlererkennung}}\\
\newline
Eine Möglichkeit fehlerbehaftete Ware zu erkennen, kann mit Hilfe eines Vergleichs der Bilder zwischen einem möglichst makellosen Keks (Musterbild) und des zu kontrollierenden Gebäcks (Kontrollbild) erreicht werden.\\
\newline
Zunächst muss aber die Ware von der Kamera erfasst werden. Hier kann es natürlich vorkommen, dass der Rand vom Keks nicht parallel zur Bildkante verläuft (vgl. Abb. \ref{fig:keks_original}). Nun muss die Aufnahme so gedreht werden, damit beide Kekse miteinander verglichen werden können - sonst werden falsche Bereiche miteinander verarbeitet und das Ergebnis ist unbrauchbar. Je nach Fertigungsgenaugikeit kann an den Luftlöchern oder mit Hilfe einer Bounding-Box am Keksrand orientiert werden.\\
\newline
Wenn nun die Aufnahme bereit ist zum Vergleich, sollen nun die Pixelwerte beider Bilder voneinander subtrahiert werden. Dadurch lassen sich auf einfache Weiße Unterschiede erkennen. Ist beispielsweise das Kontrollbild exakt das gleiche, wie das Musterbild, so subtrahieren sich für jeden Pixel im Bild immer die selben Werte zu 0 und es ensteht ein schwarzes Bild. Ergeben sich Unterschiede weißt das resultierende Bild auf diese Stellen mit anderen Pixelwerten hin. Hierfür bieten sich  verschiedene Varianten an, z.B.:
\begin{equation}
\textbf{R} = \textbf{K} - \textbf{M}
\end{equation}
\textbf{R} $\widehat{=}$ resultirende Bildmatrix, \textbf{K} $\widehat{=}$ Bildmatrix vom Kontrollbild, \textbf{M} $\widehat{=}$ Bildmatrix vom Musterbild. \\
\newline
Wenn nach dieser Variante ein Pixelwert von \textbf{M} größer als der korrespondierende aus \textbf{K} ist, so erhält \textbf{R} einen negativen Wert für diese Stelle. Solche Werte (also $\leq$ 0) können dann auf 0 gesetzt werden. Dies würde aber ein Informationsverlust bedeuten. Stattdessen bleibt die Information erhalten, wenn nach der Subtraktion noch der Betrag gebildet wird:
\begin{equation}
\textbf{R} = \vert \textbf{K} - \textbf{M} \vert 
\end{equation}
Je nach Anforderung kann nun hier nach Qualitätsstufen weiter unterteilt werden, in z.B.: fehlende Zähne oder Teig, der beim Backen übergelaufen ist. Hierfür müssen dann weitere Klassifizierungsalgorithmen auf \textbf{R} angwendet werden. Üblicherweise soll aber jeder Keks mit irgendeinem Fehler aussortiert werden. Dafür bietet sich ein einfacheres Verfahren an: so kann aussortiert werden, wenn ein bestimmter Helligkeitswert überschritten worden ist, oder wenn dieser Wert eine bestimmte Anzahl überschritten wurde (mit Hilfe einer Histogrammanalyse).\\
\textbf{{\large Kamera und Umgebung}}\\
\newline
Für dieses Verfahren ist die beste Position der Kamera direkt über dem Mittelpunkt des Gebäcks, da so die meisten Fehler auf der Draufsicht erkennt werden können. Die abzufotografierende Fläche muss beleuchtet werden, um den kompletten Keks analysieren zu können. Die Lichtquelle selbst muss möglichst nah bei der Kamera platziert werden, um Schatten zu vermeiden (ggf. mehrere Lichtquellen verwenden). \\
\newline
\textbf{{\large Integration in den Fertigungsprozess}}\\
\newline
Die Kekse werden nacheinander, im Abstand von einer Aufnahmenbreite (so dass nur ein Gebäck pro Foto aufgenommen und analysiert wird), mit Hilfe eines Förderbands an der Kamera vorbeigefahren. Um die Bildverarbeitungszeit gering zu halten kann ein zusätzlicher Sensor (z.B.: Lichtschranke in Richtung der Gebäckkante) die Fotoaufnahme auslösen, wenn der Keks komplett im Bildbereich liegt. Dieser Sensor kann auch zur Synchronisation verwendet werden, in dem er die Warenauflage auf das Förderband auslöst. Durch ihn muss dann nicht erst überprüft werden ob das Gebäck komplett oder nur teilweise auf der Aufnahme zu sehen ist. Fehlerbehaftete Kekse können letztlich mit Hilfe von pneumatischen Modulen vom Band geschoben werden.
\newpage
\section{Teller sind ein weites Feld (2 Bonuspunkte)}




%-------Text-End------------------------------------------
\end{document}

