\documentclass{ezb}
\usepackage[]{todonotes}
\usepackage{amsmath}
\usepackage{gensymb}
\usepackage{wrapfig}
\usepackage{longtable}
\usepackage{amssymb}
\usepackage[colorlinks,        	% Links ohne Umrandungen in zu wählender Farbe
   linkcolor=black,   			% Farbe interner Verweise
   filecolor=black,   			% Farbe externer Verweise
   citecolor=black    			% Farbe von Zitaten
]{hyperref}
\usepackage{booktabs}

\renewcommand{\thesubsection}{\alph{subsection}}
\begin{document}

% \maketitle{Nummer}{Abgabedatum}{Tutor-Name}{Gruppennummer}
%           {Teilnehmer 1}{Teilnehmer 2}{Teilnehmer 3}
\maketitle{05.06.15}{Udo Frese}{1}{Annika Ofenloch - 2992807 - ofenloch@uni-bremen.de}{Frank Ihle - 3010158 - fihle@uni-bremen.de}{Simon Schirrmacher - 4000884 - simons@informatik.uni-bremen.de}{Noshaba Cheema - ncheema@uni-bremen.de}

%-------Text-Start------------------------------------------
\section{Ballspiele I (10 Punkte)}

\section{Tag des Modellbaus (4 Punkte)}
Unter Verwendung der Bildverarbeitung soll ein Zeppelinmodell autonom eine vordefinierte Route über einer Wiese fliegen.\\
\\
Um die Orientierung des Zeppelins auf der Wiese zu ermöglichen, werden Markierungen auf der Wiese ausgelegt, welche als Wegpunkte zu interpretieren sind. Die Markierungen haben die Form eines Pfeils und geben somit Aufschluss über die zu fliegende Richtung.Um diese erfassen zu können, wird unter dem Zeppelin eine Kamera angebracht, die senkrecht nach unten filmt.\\ 
\linebreak
Für den Start wird eine besondere Markierung in Form eines Kreuzes eingesetzt. Zu Beginn des Rundflugs befindet sich der Zeppelin direkt auf dem Kreuz und schwebt beim Start senkrecht in die Höhe. Anhand der Größe des Kreuzes, welche mit zunehmender Höhe immer weiter abnimmt, kann der Zeppelin feststellen, ob bereits die gewünschte Höhe erreicht wurde.\\ 
\linebreak
Ist diese Höhe erreicht, orientiert sich der Zeppelin an einem Arm des Startkreuzes, der deutlich Länger ist, um die zu fliegende Richtung zu bestimmen. Jede durch die Markierungen definierte Richtung zeigt auf den Mittelpunkt (Schwerpunkt) der nächsten Markierung. Die Markierungen, welche die Wegpunkte beschreiben, sind einfarbig gefärbt und es ist eine bestimmte, sich wiederholende, Farbfolge definiert, wodurch bei mehreren erkannten Markierungen die nächst folgende ausgewählt werden kann.\\ 
\linebreak
Zudem sind sie so geformt, dass nicht die gesamte Markierung im Bild erfasst sein muss, um die vorgegebene Richtung zu erfassen. Um dies zu erreichen, sind Anfang und Ende des Pfeils so geformt, dass diese eindeutig erkennbar sind.\\ 
\linebreak
Wird eine Markierung erkannt, so nähert sich der Zeppelin dieser Markierung bis zu einem bestimmten Abstand, indem immer wieder die Entfernung zum Bildmittelpunkt bestimmt wird. Hat der Zeppelin einen bestimmten Höchstabstand unterschritten, wird die Richtung der Markierung erfasst und der Zeppelin richtet sich aus. Anschließend fliegt der Zeppelin geradeaus, bis eine weitere Markierung erkannt wird. Solange die vorherige Markierung noch erfassbar ist, wird die Richtung angepasst, falls der Zeppelin vom Kurs abgekommen ist.\\ 
\linebreak
Für die Landung wird ein weiteres Kreuz als Markierung eingesetzt, welches jedoch gleichförmig ist. Wird dieses Kreuz erkannt, sinkt der Zeppelin senkrecht.

\section{Spargel, Bildverarbeitung und soziale Realität (1 Bonuspunkt)}
Qualitätsmerkmale, wie beispielsweise die Farbe (bzw. Verfärbung), Vollständigkeit als auch die Form, lassen sich durch Angestellte schnell kontrollieren. Es kann demnach überprüft werden, ob der Spargel von Schädlingen befallen, verfault, verschmutzt, abgebrochen, krumm oder hohl ist. Bei den Merkmalen wir Länge und Durchmesser reicht das Augenmaß nicht immer aus. Insbesondere beim Durchmesser ist es schwierig die präzise Unterscheidung in die einzelnen Güteklassen durchzuführen, da es sich hier lediglich um Abweichungen von Millimetern handelt. Der Angestellte hat demnach Schwierigkeiten zu beurteilen, ob der weiße Spargel einen Durchmesser von 8 oder 10 mm aufweist (Mindestdurchmesser bei Klasse I: 10 mm, bei Klasse II: 8 mm). Auch die Länge des Spargels kann durch Angestellte nicht immer mit bloßen Augenmaß richtig eingeschätzt werden. Sobald sich die Länge in der Nähe von Grenzwerten befindet, ist es besonders wichtig, die exakte Länge zu bestimmen. Wenn der weiße Spargel zum Beispiel 17,3 cm lang ist, zählt er zum langen Spargel und nicht mehr zum kurzen Spargel. Ein Angestellter würde dies gegebenenfalls nicht sehen und den Spargel falsch zuordnen. Es ist in solchen Fällen also wichtig, wenn sich der Angestellte nicht sicher ist, den Spargel an ein Bildverarbeitungssystem weiter zu geben. Dort kann der genaue Durchmesser als auch die Länge des Spargels ermittelt werden. Wenn die Länge hingegen nur 5 cm beträgt, wird auch der Angestellte mit bloßem Augenmaß in der Lage sein, den Spargel auszusortieren. 

%-------Text-End------------------------------------------
\end{document}

